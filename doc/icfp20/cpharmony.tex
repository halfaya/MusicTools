\section{Representations of Counterpoint and Harmony}
\label{sec:cpharmony}

Harmony refers to simultaneously sounding tones (chords) and harmonic progression
a series of chords. It is in essence the of dual of counterpoint. A
fundamental representation used in Music Tools is a grid of
\texttt{Point}s, where a \texttt{Point} represents either a pitch, a
hold of a previous pitch, or a rest.

\begin{alltt}
data Point : Set where
  tone : Pitch \(\rightarrow\) Point
  hold : Pitch \(\rightarrow\) Point
  rest : Point
\end{alltt}

Music is thus considered to be quantized to some minimal duration,
with the horizontal dimension representing time and the vertical
dimension representing pitch. A slice in the horizontal direction is a
\texttt{Melody}, represented as a vector of points of some length $n$. Similarly a slice
in the vertical direction is a \texttt{Chord}, again of some fixed length.

\begin{alltt}
data Melody (n : \(\mathbb{N}\)) : Set where
  melody : Vec Point n \(\rightarrow\) Melody n

data Chord (n : \(\mathbb{N}\)) : Set where
  chord : Vec Point n \(\rightarrow\) Chord n
\end{alltt}

One can think of the grid as several melodies in parallel of the same
duration. This is essentially counterpoint, and thus given that
name. Note the crucial use of dependent types here to ensure all
melodies have the same duration $d$. The counterpoint is specified to
have $v$ voices.

\begin{alltt}
data Counterpoint (v : \(\mathbb{N}\)) (d : \(\mathbb{N}\)): Set where
  cp : Vec (Melody d) v \(\rightarrow\) Counterpoint v d
\end{alltt}

Dually one can think of the grid as a series of chords, in another
word a harmonic progression, although this is named \texttt{Harmony}
in the library for conciseness. Again dependent types enforce every
chord has the same number of voices.

\begin{alltt}
data Harmony (v : \(\mathbb{N}\)) (d : \(\mathbb{N}\)): Set where
  harmony : Vec (Chord v) d \(\rightarrow\) Harmony v d
\end{alltt}

Converting between \texttt{Counterpoint} and \texttt{Harmony} is
simply matrix transposition. These are fundamental representations
that can be used directly for analysis, for example, and there are
functions to convert them to MIDI for sound generation.

There are many other ways to represent music, which have advantages in
certain contexts. For example in Section~\ref{sec:cp}
\texttt{PitchInterval} and \texttt{PitchInterval2} were used instead
of \texttt{Counterpoint} since they more explictly represent the
structure of the music in species counterpoint. It is straightforward
to convert between the representations, and worth using the most
appropriate type for a given situation.
