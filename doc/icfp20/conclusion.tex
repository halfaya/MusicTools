\section{Conclusion and Future Work}
\label{sec:conclusion}

We believe that functional programming, and in particular functional
programming with dependent types, is the most promising way to do software
engineering going forward. The languages and tools are perhaps not yet
as finely polished and rich as those typically used in production
software, but they are fundamentally far more powerful, and thus once
the infrastructure (and training of programmers) catches up should
become the languages of choice.

Music is an especially valuable domain in which to explore these
ideas. We have found it to be a microcosm of issues that arise
everywhere in software engineering: modularity, composability,
correctness, and data issues such as handling equivalent formulations of
the same data as well slightly different formulations. If we can
demonstrate that functional programming with dependent types works
well in the domain of music, the same techniques can be used in
software in general.

Data issues are particularly prevalent at API boundaries between
systems, which may internally represent the data in different formats
(satisfying modularity and abstraction) but then rely on tedious and
potentially error-prone conversions between the formats. In music
we see this in wanting to work sometimes with absolute pitch and others
with relative pitch (a pair of an octave and a pitch within that
octave), or choosing between
a pair of pitches and a \texttt{PitchInterval}. Also we would like to
reuse a function that just works on a pitch to work on a note, which
consists of a pitch plus a duration.

For now we have explicity converted between equivalent representations
and explictly lifted functions, to get a feel for the amount of work
required.  However in the world of dependent types there are known
research techniques for handling these cases, namely transporting
across equivalences as in Homotopy Type Theory~\citep{hottbook} and
ornaments~\citep{dagand-ornaments}. Cubical
Agda~\citep{vezzosi-cubical}, although still early in development,
should be an excellent tool for exploring the extent to which these
research ideas can be applied in a practical context.

Aside from these more fundamental ideas there is simply the work of
further extending the formalization of music theory to encompass
harmonic analysis, more advanced counterpoint, voice leading, and even
composition~\citep{schoenberg-fundamentals}. The results so far have
been encouraging, but much more needs to be done to find the right
abstractions to express the rules in the simplest, clearest and most
composeable way possible.
We expect the world of music theory to be worthy of a long exploration.

