%% For double-blind review submission, w/o CCS and ACM Reference (max submission space)
\documentclass[sigplan,review]{acmart}\settopmatter{printfolios=true,printccs=false,printacmref=false}
%% For double-blind review submission, w/ CCS and ACM Reference
%\documentclass[acmsmall,review,anonymous]{acmart}\settopmatter{printfolios=true}
%% For single-blind review submission, w/o CCS and ACM Reference (max submission space)
%\documentclass[acmsmall,review]{acmart}\settopmatter{printfolios=true,printccs=false,printacmref=false}
%% For single-blind review submission, w/ CCS and ACM Reference
%\documentclass[acmsmall,review]{acmart}\settopmatter{printfolios=true}
%% For final camera-ready submission, w/ required CCS and ACM Reference
%\documentclass[acmsmall]{acmart}\settopmatter{}

%% Journal information
%% Supplied to authors by publisher for camera-ready submission;
%% use defaults for review submission.
\acmJournal{PACMPL}
\acmVolume{1}
\acmNumber{CONF} % CONF = POPL or ICFP or OOPSLA
\acmArticle{1}
\acmYear{2018}
\acmMonth{1}
\acmDOI{} % \acmDOI{10.1145/nnnnnnn.nnnnnnn}
\startPage{1}

%% Copyright information
%% Supplied to authors (based on authors' rights management selection;
%% see authors.acm.org) by publisher for camera-ready submission;
%% use 'none' for review submission.
\setcopyright{none}
%\setcopyright{acmcopyright}
%\setcopyright{acmlicensed}
%\setcopyright{rightsretained}
%\copyrightyear{2018}           %% If different from \acmYear

%% Bibliography style
\bibliographystyle{ACM-Reference-Format}
%% Citation style
%% Note: author/year citations are required for papers published as an
%% issue of PACMPL.
\citestyle{acmauthoryear}   %% For author/year citations

%% Some recommended packages.
\usepackage{booktabs}   %% For formal tables:
                        %% http://ctan.org/pkg/booktabs
\usepackage{subcaption} %% For complex figures with subfigures/subcaptions
                        %% http://ctan.org/pkg/subcaption


\begin{document}

%% Title information
\title{Demo: Counterpoint by Construction}
%\subtitle{Functional Pearl}

%% Author information
%% Contents and number of authors suppressed with 'anonymous'.

%% Author with single affiliation.
\author{Youyou Cong}
\affiliation{
  \institution{Tokyo Institute of Technology}            %% \institution is required
  \city{Tokyo}
  \country{Japan}
}
\email{cong@c.titech.ac.jp}          %% \email is recommended

\author{John Leo}
\affiliation{
  \institution{Halfaya Research}            %% \institution is required
  \city{Bellevue}
  \state{WA}
  \country{USA}
}
\email{leo@halfaya.org}          %% \email is recommended


%% Abstract
%% Note: \begin{abstract}...\end{abstract} environment must come
%% before \maketitle command
\begin{abstract}
Western music of the common practice period tends to loosely follow sets of rules which
were developed over time to help ensure the aethetic quality of the composition. Rules in
developed in particular for harmony (\cite{piston1987harmony}) and counterpoint (\cite{fux1965study}),
and continue to be taught to music students
both to to understand the music of that time and because they continue to provide a foundation for
modern art and popular music. To help both analyze and synthesize
tonal music, it is worthwhile to attempt to encode these rules into a programming
language, with functional programming languages being particularly wells-suited to this task.
Recently quite a bit of work has been done in Haskell (\cite{fmmh, hihseufha, faamh, hafha, fghm})
to encode the rules of harmony, but little work seems to have been done with counterpoint (\cite{mezzo}).
Furthermore Haskell's type system, lacking full dependent types, is not strong enough to encode
the required logic in a natural way, and requires using a number of extensions.

In this demonstration of work in progress, we present Music Tools (\cite{MusicTools}), a library of
small tools that can be combined functionally to help analyze and synthesize music. The library is
written in the dependently typed language Agda (with Haskell used for I/O), which allows a simple
and natural encoding of rules. We demonstrate its use in writing species counterpoint following
Fux (\cite{fux1965study}), in which type system ensures by construction that the rules are followed.
We show how this both aids human composition and allows for computer-generated creation of correct
counterpoint. We also contrast this work to recent use of machine learnring to generated natural-sounding
counterpoint which does not necessarily follow specific rules (\cite{CounterpointByConvolution}). Finally
we indicate future planned work including extension to handle functional harmony.
\end{abstract}


%% 2012 ACM Computing Classification System (CSS) concepts
%% Generate at 'http://dl.acm.org/ccs/ccs.cfm'.
\begin{CCSXML}
<ccs2012>
<concept>
<concept_id>10011007.10011006.10011008</concept_id>
<concept_desc>Software and its engineering~General programming languages</concept_desc>
<concept_significance>500</concept_significance>
</concept>
<concept>
<concept_id>10003456.10003457.10003521.10003525</concept_id>
<concept_desc>Social and professional topics~History of programming languages</concept_desc>
<concept_significance>300</concept_significance>
</concept>
</ccs2012>
\end{CCSXML}

\ccsdesc[500]{Software and its engineering~General programming languages}
\ccsdesc[300]{Social and professional topics~History of programming languages}
%% End of generated code

%% Keywords
%% comma separated list
\keywords{keyword1, keyword2, keyword3}  %% \keywords are mandatory in final camera-ready submission


%% \maketitle
%% Note: \maketitle command must come after title commands, author
%% commands, abstract environment, Computing Classification System
%% environment and commands, and keywords command.
\maketitle


%% Acknowledgments
\begin{acks}                            %% contents suppressed with 'anonymous'
  Thanks to the participants of the Tokyo Agda Implmentors' Meeting, especially Ulf Norell and Jesper Cockx,
  for many helpful suggestions that improved our Agda code.
\end{acks}


%% Bibliography
\bibliography{farm-abstract.bib}

\end{document}
