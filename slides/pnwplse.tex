 \documentclass{beamer}
\usefonttheme{professionalfonts}
\usepackage{agda}
\usepackage{catchfilebetweentags}
\usetheme{CambridgeUS}
\usecolortheme{seahorse}

\usepackage{amssymb}
\usepackage{bbm}
\usepackage[greek,english]{babel}

\beamertemplatenavigationsymbolsempty

\title{Musical Ornaments}
\author{John Leo}
\institute{Halfaya Research}
\date{May 14, 2018}
 
\begin{document}
 
\frame{\titlepage}
 
\begin{frame}\frametitle{Outline}
\begin{itemize}
\item A look toward the future
\item Music Tools
\item Equivalences
\item Ornaments
\end{itemize}

\bigskip 
Sources:
\begin{itemize}
\item {\tt https://github.com/halfaya/MusicTools}
\end{itemize}
\end{frame}

\begin{frame}\frametitle{Robert Harper}
\begin{quote}
Eventually all the arbitrary programming languages are going to be just swept away with the oceans,
and we will have the permanence of constructive, intuistionistic type theory as the master theory
of computation---without doubt, in my mind, no question.  So, from my point of view---this is a personal
statement---working in anything else is a waste of time.
\end{quote}

CMU Homotopy Type Theory lecture 1, 52:56--53:20.
\end{frame}

\begin{frame}\frametitle{Music Representation}
\ExecuteMetaData[../agda/latex/Pnwplse.tex]{pitch}
\ExecuteMetaData[../agda/latex/Pnwplse.tex]{duration}
\ExecuteMetaData[../agda/latex/Pnwplse.tex]{note}
\ExecuteMetaData[../agda/latex/Pnwplse.tex]{music}
\end{frame}

\begin{frame}\frametitle{Conclusion}
Goes here.
\end{frame}

\end{document}
