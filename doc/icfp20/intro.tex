\section{Introduction}
\label{sec:intro}

Edgar Var\`{e}se describes music as ``organized sound'', and
throughtout history cultures have created and applied systems of rules
and guidelines to govern the music they create, most notably the
Common Practice Period of Western music spanning the 17th to early
20th centuries. These rule systems are seldom absolute, and indeed
deliberate breaking the rules is often part of the aesthetic, but they
roughly constrain the music they apply to and give it a common form
and sound.

Artists and theoreticians have attempted to capture and codify these
rule systems, always informally in natural language, and typically
accompanied by examples from the existing literature. The intent is
both to analysize existing music and then to use these principles to
guide the creation of new music, in other words for synthesis.

Starting in the 20th century computers have become ubiquitous in
music in every area including sound sythesis, composition and
production (\citep{roads-tutorial}).

[more here]

