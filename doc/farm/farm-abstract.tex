%% For double-blind review submission, w/o CCS and ACM Reference (max submission space)
\documentclass[sigplan,10pt,screen]{acmart}
\settopmatter{printfolios=true, printccs=true, printacmref=true}
%% For double-blind review submission, w/ CCS and ACM Reference
%\documentclass[acmsmall,review,anonymous]{acmart}\settopmatter{printfolios=true}
%% For single-blind review submission, w/o CCS and ACM Reference (max submission space)
%\documentclass[acmsmall,review]{acmart}\settopmatter{printfolios=true,printccs=false,printacmref=false}
%% For single-blind review submission, w/ CCS and ACM Reference
%\documentclass[acmsmall,review]{acmart}\settopmatter{printfolios=true}
%% For final camera-ready submission, w/ required CCS and ACM Reference
%\documentclass[acmsmall]{acmart}\settopmatter{}

%%% The following is specific to FARM '19 and the paper
%%% 'Demo: Counterpoint by Construction'
%%% by Youyou Cong and John Leo.
%%%
\setcopyright{rightsretained}
\acmPrice{}
\acmDOI{10.1145/3331543.3342578}
\acmYear{2019}
\copyrightyear{2019}
\acmISBN{978-1-4503-6811-7/19/08}
\acmConference[FARM '19]{Proceedings of the 7th ACM SIGPLAN International Workshop on Functional Art, Music, Modeling, and Design}{August 23, 2019}{Berlin, Germany}
\acmBooktitle{Proceedings of the 7th ACM SIGPLAN International Workshop on Functional Art, Music, Modeling, and Design (FARM '19), August 23, 2019, Berlin, Germany}

%% Bibliography style
\bibliographystyle{ACM-Reference-Format}
%% Citation style
%% Note: author/year citations are required for papers published as an
%% issue of PACMPL.
\citestyle{acmauthoryear}   %% For author/year citations

%% Some recommended packages.
\usepackage{booktabs}   %% For formal tables:
                        %% http://ctan.org/pkg/booktabs
\usepackage{subcaption} %% For complex figures with subfigures/subcaptions
                        %% http://ctan.org/pkg/subcaption
\usepackage{alltt}

\begin{document}

%% Title information
\title{Demo: Counterpoint by Construction}
%\subtitle{Functional Pearl}

%% Author information
%% Contents and number of authors suppressed with 'anonymous'.

%% Author with single affiliation.
\author{Youyou Cong}
\affiliation{
  \institution{Tokyo Institute of Technology}            %% \institution is required
  \city{Tokyo}
  \country{Japan}
}
\email{cong@c.titech.ac.jp}          %% \email is recommended

\author{John Leo}
\affiliation{
  \institution{Halfaya Research}            %% \institution is required
  \city{Bellevue}
  \state{WA}
  \country{USA}
}
\email{leo@halfaya.org}          %% \email is recommended


%% Abstract
%% Note: \begin{abstract}...\end{abstract} environment must come
%% before \maketitle command
\begin{abstract}
We present \emph{Music Tools}, an Agda library for analyzing 
and synthesizing music.  
The library uses dependent types to simplify encoding of 
music rules, thus improving existing approaches based on 
simply typed languages.  
As an application of the library, we demonstrate an 
implementation of first-species counterpoint, where we use
dependent types to constrain the motion of two parallel
voices.
\end{abstract}


%% 2012 ACM Computing Classification System (CSS) concepts
%% Generate at 'http://dl.acm.org/ccs/ccs.cfm'.
\begin{CCSXML}
<ccs2012>
<concept>
<concept_id>10010405.10010469.10010475</concept_id>
<concept_desc>Applied computing~Sound and music computing</concept_desc>
<concept_significance>500</concept_significance>
</concept>
<concept>
<concept_id>10011007.10011006.10011008.10011009.10011012</concept_id>
<concept_desc>Software and its engineering~Functional languages</concept_desc>
<concept_significance>500</concept_significance>
</concept>
</ccs2012>
\end{CCSXML}

\ccsdesc[500]{Applied computing~Sound and music computing}
\ccsdesc[500]{Software and its engineering~Functional languages}
%% End of generated code

%% Keywords
%% comma separated list
\keywords{counterpoint, dependent types, Agda}  %% \keywords are mandatory in final camera-ready submission


%% \maketitle
%% Note: \maketitle command must come after title commands, author
%% commands, abstract environment, Computing Classification System
%% environment and commands, and keywords command.
\maketitle

\section{Introduction}

Western music of the common practice period tends to loosely
follow sets of rules, which were developed over time to ensure
the aesthetic quality of the composition. 
Among such rules, those for harmony \citep{piston1987harmony}
and counterpoint (harmonically interdependent melodies) 
\citep{fux1965study} are particularly fundamental and continue 
to be taught to music students, not only as a means to understand 
the music of that period, but also as a foundation for modern art 
and popular music.

To analyze and synthesize tonal music, researchers have
attempted to encode these rules into programming languages.
Functional programming languages seem
ideally suited for this task, and in particular, those with a static
type system can further guarantee that \emph{well-typed music 
does not sound wrong}.
In the past decade, Haskell has been a popular choice for encoding
the rules of harmony \citep{fmmh, hihseufha, faamh, hafha, fghm}
as well as counterpoint \citep{szamozvancev2017well}.
An interesting observation is that, many of the existing encodings
rely on some form of \emph{dependent types}, i.e.,  
types that depend on terms. 
While Haskell is not a dependently typed language, it has various 
language extensions (such as GADTs \citep{cheney2002lightweight} 
and singleton types \citep{eisenberg2013dependently}) for 
simulating dependent types, and as shown by previous studies,
the extended type system is expressive enough for 
a wide class of music applications.
On the other hand, the need for simulating dependent types 
can also result in code duplication, making the implementation 
less elegant \citep{monnier2010singleton}.
This motivates us to explore music programming in a language
with intrinsic support for dependent types.

We present Music Tools\footnote{\url{https://github.com/halfaya/MusicTools}},
a library of small tools that can be combined functionally to 
help analyze and synthesize music.
To allow simple and natural encoding of rules, we build our 
library in Agda \citep{norellphd}, a functional language with full dependent types.
As an application of the library, we demonstrate an
implementation of species counterpoint, based on the rules 
given by \citet{fux1965study}.
Thanks to Agda's rich type system, we can express 
the rules in a straightforward manner, and thus ensure by 
construction that well-typed counterpoint satisfies all the 
required rules.


\section{The Music Tools Library}

The goal of Music Tools is to provide a core collection of types and functions
that can be used to easily build applications to analyze and synthesize music.
It is intended to evolve into a dependently-typed replacement for Euterpea
\citep{hudak2018haskell}, and borrows many ideas from that library.
Currently, Music Tools is restricted to the chromatic scale for simplicity, 
and like Euterpea is designed to work well with MIDI.

The fundamental type for melody is \texttt{Pitch}, which is just a natural number with $0$
representing the lowest expressible pitch (it is intended to correspond to MIDI note 0, but
the interpretation can change depending upon the application). One can also
express pitch as a pair of an octave and relative pitch within the octave, which is
more convenient for some applications, and convert between this representation and
absolute pitch. One can then prove that converting back and forth is the identity function.

For rhythm the fundamental type is \texttt{Duration}, also a natural number, which represents
some unspecified unit of time (when the music is played the unit is then specified).
A combination of a pitch and a duration gives a \texttt{Note}, and notes in turn are
made into \texttt{Music} via sequential and parallel composition, as in Euterpea.

To play music on synthesizers, one can convert it to MIDI by calling 
Haskell MIDI libraries via Agda's Haskell FFI. 
Details can be found in the GitHub repository.

\section{Application: First-Species Counterpoint}

We now explain how to implement the rule system of 
first-species counterpoint\footnote{The code is available at \\
\url{https://github.com/halfaya/MusicTools/blob/master/agda/Counterpoint.agda}.}.
In first-species counterpoint, one starts with a base melody 
(the \emph{cantus firmus}), and constructs a counterpoint 
melody note-by-note in the same rhythm.
The two voices are represented as a list of pitch-interval pairs,
where intervals must not be dissonant (2nds, 7ths, or 4ths).

\begin{alltt}
data IntervalQuality : Set where
  min3  : IntervalQuality
  maj3  : IntervalQuality
  per5  : IntervalQuality
  min6  : IntervalQuality
  maj6  : IntervalQuality
  per8  : IntervalQuality
  min10 : IntervalQuality
  maj10 : IntervalQuality

PitchInterval : Set
PitchInterval = Pitch \(\times\) IntervalQuality
\end{alltt}

In addition, it is prohibited to move from any interval to
a perfect interval (5th or octave) via parallel or similar motion.
Therefore, we define a predicate that checks whether a motion 
is allowed or not.

\begin{alltt}
motionOk : (i1 : Interval)
           (i2 : Interval) \(\rightarrow\) Set
motionOk i1 i2 with motion i1 i2
         | isPerfectInterval i2
motionOk i1 i2 | contrary | \_     = \(\top\)
motionOk i1 i2 | oblique  | \_     = \(\top\)
motionOk i1 i2 | parallel | false = \(\top\)
motionOk i1 i2 | parallel | true  = \(\bot\)
motionOk i1 i2 | similar  | false = \(\top\)
motionOk i1 i2 | similar  | true  = \(\bot\)
\end{alltt}

The last requirement is that the music must end with a cadence,
which is a final motion from the 2nd or 7th degree to the tonic 
(1st degree). 
We impose this requirement by declaring two cadence constructors 
as the base cases of counterpoint (note that the final interval of
the cadence is always \texttt{(p\ ,\ per8)} and hence is not explicitly
specified).
Thus, we arrive at the following datatype for well-typed counterpoint\footnote{
For readability, we have omitted explicit conversions 
from \texttt{PitchInterval} (which ensures the interval is not dissonant) 
to the general \texttt{Interval}.}.

\begin{alltt}
data FirstSpecies : PitchInterval \(\rightarrow\)
                    Set where
  cadence2 : (p : Pitch) \(\rightarrow\)
    FirstSpecies (transpose (+ 2) p , maj6)
  cadence7 : (p : Pitch) \(\rightarrow\)
    FirstSpecies (transpose -[1+ 0 ] p , min10)
  \_::\_ : (pi : PitchInterval) \(\rightarrow\)
         \{pj : PitchInterval\} \(\rightarrow\)
         \{\_ : motionOk pi pj\} \(\rightarrow\)
         FirstSpecies pj \(\rightarrow\)
         FirstSpecies pi
\end{alltt}

\noindent Observe that \texttt{motionOk} is an implicit argument of 
the \texttt{\_::\_} constructor, here denoted by the curly braces. 
The argument can be resolved automatically by the type checker,
hence there is no need to manually supply this proof. 

Now we can write valid first-species counterpoint as in the 
example below.

\begin{alltt}
example : FirstSpecies (g 4 , per8)
example = 
  (g 4 , per8) :: (c 5 , maj10) ::
  (c 5 , per8) :: (c 5 , maj10) ::
  (e 5 , min10) :: (g 5 , per8) ::
  (cadence2 (c 6))
\end{alltt}


\section{Future Work}

We intend to expand the Music Tools library to include
richer functionality for analysis and synthesis. We are
particularly interested in expressing the work done in Haskell
on functional harmony in the library, to see how much
dependent types can simplify the representation.

Music is a rich yet circumscribed domain in which
issues of equivalence \citep{tabareau2017equivalences}
and ornamentation \citep{dagand2017essence} naturally arise.
For example, a
musical score can be interpreted either horizontally (counterpoint) or
vertically (harmony), and it is important to be able to seamlessly
convert between these representations. Similarly, one may wish to treat
pitch or rhythm separately as well as combine them, which is a natural
application of ornamentation. We plan to examine the extent to which this
research can be put to use in a practical domain.

For counterpoint specifically, we intend to represent higher species 
counterpoint by extending the rules, 
and to explore automatic generation of species counterpoint.
It would be interesting to compare our correct-by-construction counterpoint
with that created by machine learning \citep{CounterpointByConvolution}, 
which does not have correctness guarantees.

%% Acknowledgments
\begin{acks}                            %% contents suppressed with 'anonymous'
 The authors would like to thank the participants of the Tokyo Agda 
 Implementors' Meeting, especially Ulf Norell and Jesper Cockx,
 for many helpful suggestions that improved our Agda code.
 We also thank the anonymous reviewers for their thoughtful feedback,
 which improved our presentation.
\end{acks}


%% Bibliography
\bibliography{farm-abstract.bib}

\end{document}